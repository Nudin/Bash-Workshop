\documentclass{beamer}
\usepackage[utf8]{inputenc}
\usepackage[german]{babel}
\usepackage{amsmath}
\usepackage{amsfonts}
\usepackage{amssymb}
\usepackage{framed, color}
\usepackage{listings}

\definecolor{dkgreen}{rgb}{0,0.6,0}
\definecolor{gray}{rgb}{0.5,0.5,0.5}
\definecolor{mauve}{rgb}{0.58,0,0.82}


\lstset{
  language=bash,
  numbers=left, 
  stepnumber=1, 
  numberstyle=\tiny\color{gray},
  showspaces=false,
  showstringspaces=false,
  keywordstyle=\color{blue},          % keyword style
  commentstyle=\color{dkgreen},       % comment style
  stringstyle=\color{mauve},         % string literal style
  basicstyle=\footnotesize, 
  frame=single,
  morekeywords={elif}
  }

\usetheme{Warsaw}  %% Themenwahl

\title{Bash-Workshop}
\author{Michael F. Schönitzer}
\date{14. Juli 2012}

\begin{document}
\maketitle
 
%\frame{\tableofcontents[pausesections]}
\tableofcontents

\section{Grundlagen}

\subsection{Geschichte \& Hintergründe}
\begin{frame} %%Eine Folie
  \frametitle{Geschichte \& Hintergründe} %%Folientitel
  \begin{Definition} %%Definition
    Eine Unix-Shell ist die traditionelle Benutzerschnittstelle unter Unix oder unixoiden Computer-Betriebssystemen. Die Shell ist ein Interpreter.
  \end{Definition}
  Geschichtlich:
  \begin{itemize}
   \item Thompson-Shell (osh) 1971-1979 
   \item Bourne-Shell (sh) ab 1978
   \item C-Shell (csh) ab 1979
  \end{itemize}
  
\end{frame}

\begin{frame}
  \frametitle{Geschichte \& Hintergründe} %%Folientitel
  
  \begin{itemize}
   \item Korn Shell (ksh)
   \item Z-Shell (zsh)
   \item Bourne Again Shell (bash)
   \begin{itemize}
    \item Kompatibel zur Bourne-Shell (sh)
    \item Die meisten Funktionen der Korn-Shell
    \item Teile der C-Shell (csh)-Syntax
    \item Weitere Funktionen
    \item Standard-Shell auf den meisten Systemen
   \end{itemize}
  \end{itemize}
\end{frame}

\subsection{Grundlagen}
\begin{frame}[fragile,label=code]
 \frametitle{Grundlagen}
 
 \begin{itemize}
  \item Alles was auf der Kommandozeile eingetippt wird, kann auch in Dateien geschrieben werden und umgekehrt!
  \item Programme, Bash built-in commands und Syntax
  \item Hello-World:
  \begin{lstlisting}
    #!/bin/bash
    
    # Kommentar
    echo "Hello World"
  \end{lstlisting}
  \item She-Bang, Kommentar, echo, Anführungszeichen
  \item Semikon ist äquivalent zu Zeilenumbruch
  \item Dateiendung: .sh
 \end{itemize}
\end{frame}

\begin{frame}
 \frametitle{Nullte Ausprobier-Runde}
 \begin{center}
 Los!
 \end{center}
\end{frame}

\begin{frame}[fragile]
 \frametitle{Programme, Argumente, Datenströme}
 
 \only<1,2>{
 Starten eines Programms auf der Commandline:
 \begin{semiverbatim}
 \$ /pfad/zu/programm/programmname Arg Arg2 ...
 \end{semiverbatim}
 }
 \only<2>{Pfad zum Programm kann entfallen wenn dieser im PATH enthalten. Arbeitsverzeichnis ist dort nicht, daher starten von Skripten mit {\tt./skriptname}!}
  
 \begin{itemize}[<+->]
  \item Argumente
  \item Drei Standard-Datenströme:
   \begin{itemize}[<+->]
     \item Standard-Out, Stdout (\#1)
     \item Standard-In, Stdin (\#0)
     \item Standard-Error, Stderr (\#2)
   \end{itemize}
 \item Rückgabewert:
   \begin{itemize}
       \item Null falls Programm erfolgreich
       \item Anderer Wert je nach Fehler und Programm (Undefiniert!)
   \end{itemize}
 \end{itemize}
\end{frame}

\subsection{Datenströme}
\begin{frame}
\frametitle{Arbeiten mit Datenströmen}
 \begin{itemize}
  \item Pipe: leiten Stdout von einem Programm nach Stdin eines anderen Programms um\\
   {\tt\small \$ echo abc | grep a}
  \item Umleiten in Datei\\
   {\tt\small \$ echo abc 1> Dateiname}\\
   {\tt\small \$ echo abc > Dateiname}\\
   {\tt\small \$ echo abc >> Dateiname}
  \item Umleiten in anderen Stream\\
   {\tt\small \$ echo abc 2>\&1 Dateiname}
  \item Wilde Kombinationen möglich!\\
   {\tt\small strace echo a 2>\&1 | grep open }  Kurzform: {\tt |\&} \\
   {\tt\small \$ find . -name '*.html' 2>\&1 1> gefundene.txt | less }
  \item In Datei schreiben und auf Stdout anzeigen: \\
   {\tt\small \$ echo abc | tee datei }
 \end{itemize}
\end{frame}

\begin{frame}[fragile]
 \frametitle{Kommando-Substitution}
 Wird ersetzt mit der Ausgabe (Stdout) eines Programms
 \begin{lstlisting}
   mplayer $(cat filelist)
   mplayer `cat filelist`
 \end{lstlisting}
\end{frame}

\subsection{Einige wichtige Programme \& Praktischer Teil I}
\begin{frame}[<+->]
 \frametitle{Einige wichtige Programme I}
 \begin{itemize}
   \item echo -- Schreibt Argumente auf Stdout
     \begin{itemize}
       \item -e -- Interpretiert Kodierte Sonderzeichen
       \item -n -- Deaktiviert Zeilenvorschub am Ende
     \end{itemize}
   \item cat -- liest (mehrere) Dateien und gibt sie auf Stdout aus
   \item grep -- Sucht (Stdin oder Datei) zeilenweise nach Ausdruck und gibt Zeilen mit Treffer aus (Stderr)
     \begin{itemize}
       \item -v -- gibt Zeilen ohne Treffer aus
       \item -i -- ignoriert Groß-Kleinschreibung
       \item -q -- gibt nichts aus (Setzt nur Rückgabewert)
       \item ... -- siehe man
     \end{itemize}
   \item sort -- Sortiert Zeilen in Datei oder Stdin (alphabetisch)
   \item date -- Zeigt Datum und Uhrzeit an
   \item tr -- Ersetzt Zeichen durch andere oder löscht sie (-d), …
 \end{itemize}
\end{frame}

\begin{frame}
 \frametitle{Erste Ausprobier-Runde}
 \begin{center}
 Los!
 \end{center}
\end{frame}

\section{Elemente einer Programmiersprache}
\subsection{Variablen}
\begin{frame}[fragile]
 \frametitle{Variablen}
 
 \begin{lstlisting}
 a="Hallo Welt"
 b=7
 c="a ist ${a} und b ist $b"
 echo $c
 # Ausgabe: a ist Hallo Welt und b ist 7
 \end{lstlisting}
 
 \begin{alertblock}{Achtung!}
  \begin{itemize}
   \item Keine Variablentypen!
   \item Rechnen und Vergleichen nicht direkt möglich!
   \item Keine Interpretation in einfachen Anführungszeichen
  \end{itemize}
 \end{alertblock}
\end{frame}

\begin{frame}[<+->]
 \frametitle{Besondere Variablen}
 
 \begin{itemize}
  \item Übergebene Argumente: \tt\$1, \$2, \$3 ...
  \item Alle Argumente: \tt\$* oder \$@
  \item Anzahl der Argumente: \tt\$\#
  \item Rückgabewert des letzten Kommandos \tt\$?
  \item Prozess-ID der Shell: \tt\$\$
  \item Prozess-ID der letzten Unter-Shell: \tt\$!
  \item Name des Shellskipts: \tt\$0
 \end{itemize}
\end{frame}

\begin{frame}
 \frametitle{Test}
 
 \begin{itemize}
  \item Kommando zum Testen von Dateien und Vergleichen von Werten
  \item Unterscheidet zwischen Strings und Integers
  \item Kurzschreibweise: {\tt [ ... ]}
  \item Strings:
  \begin{itemize}
   \item String nicht leer: {\tt [ -n STRING ]}
   \item Strings gleich: {\tt [ STRING1 = STRING2 ]}
   \item Strings nicht gleich: {\tt [ STRING1 != STRING2 ]}
  \end{itemize}
 \item Integers:
 \begin{itemize}
  \item a kleiner (less than) b: {\tt [ a -lt b ]}
  \item a kleinergleich (less equal) b: {\tt [ a -le b ]}
  \item a gleich (equal) b: {\tt [ a -eq b ]}
 \end{itemize}
 \item Dateien:
 \begin{itemize}
  \item Datei existiert: {\tt [ -e Dateiname ]}
  \item Datei existiert und ist Ordner: {\tt [ -d Dateiname ]}
  \item Datei existiert und ist reguläre Datei: {\tt [ -f Dateiname ]}
 \end{itemize}
 \end{itemize}
\end{frame}

%\subsection{Rechnen}
\begin{frame}[fragile]
 \frametitle{Rechnen mit Variablen}
 
 \begin{itemize}
  \item Rechnen mit Kommando let
  \item Kurzform (( ... ))
  \item Gibt nichts auf stdout aus!
  \item Kann auch vergleichen.
  \item Rechnet nur mit Integers!
 \end{itemize}
 \begin{lstlisting}
  x=7; y=3
  ((x=x*2+y)) # x=17
  ((y++))     # y=4
  ((y==3))    # false
  echo $((4+7))
 \end{lstlisting}
\end{frame}

\subsection{Programmsteuerung}
\begin{frame}[fragile]
 \frametitle{Programmsteuerung: If-Else} 
  \begin{lstlisting}
  if <Bedingung>; then
    <...>
  elif <Bedingung>; then
    <...>
  else
    <...>
  fi
  \end{lstlisting}
\end{frame}

\begin{frame}[fragile]
 \frametitle{Programmsteuerung: For-Schleife} 
   \begin{lstlisting}
    for ((<Ausdruck>; <Ausdruck>; <Ausdruck>)) ; do
      <Kommandos>
    done
  \end{lstlisting}
\end{frame}

\begin{frame}[fragile]
 \frametitle{Programmsteuerung: For-In-Schleife} 
  \begin{lstlisting}
    for <Variablenname> in <Liste> ; do
      <Kommandos>
    done
  \end{lstlisting}
\end{frame}

\begin{frame}[fragile]
 \frametitle{Programmsteuerung: While-Schleife} 
  \begin{lstlisting}
  while <Kommandos>; do
   <Kommandos>
  done
  \end{lstlisting}
\end{frame}

\begin{frame}[fragile]
 \frametitle{Programmsteuerung: Until-Schleife} 
  \begin{lstlisting}
  until <Kommandos>; do
   <Kommandos>
  done
  \end{lstlisting}
\end{frame}

\begin{frame}
 \frametitle{break \& continue}
  \begin{itemize}
   \item {\tt break} -- Beendet eine Schleife
   \item {\tt continue} -- Springt zum nächsten Schleifendurchlauf
  \end{itemize}
\end{frame}

\begin{frame}[fragile]
 \frametitle{Programmsteuerung: Case} 
  \begin{lstlisting}
  case <Wort> in
    Muster) <Kommandos> ;;
    Muster) <Kommandos> ;;
  esac
  \end{lstlisting}
\end{frame}

\begin{frame}[fragile]
 \frametitle{Kontrollstrukturen und Sreams}
 
 Kontrollstrukturen (if, for, while…) geben Streams weiter.
 \medskip 
 
 Gebräuchliches Beispiel:
 \begin{lstlisting}
  while read line
    echo $line;
  done < myfile
 \end{lstlisting}
 
 Verrücktes Beispiel:
 \begin{lstlisting}
 cat file | if [ $silent -eq 1 ]; then > /dev/null
     else cat; fi
 \end{lstlisting}
\end{frame}

\subsection{Arrays}
\begin{frame}[fragile]
 \frametitle{Arrays}
 \begin{lstlisting}
  array=(01 02 03)
  array[17]=18
  echo $array           # 01
  echo ${array[0]}      # 01
  echo ${array[1]}      # 02
  echo ${array[*]}      # 01 02 03 17
  echo ${#array[*]}     # 3
 \end{lstlisting}
 
 Assoziatives array:
 \begin{lstlisting}
  declare -A aarray
  aarray[hund]=waldi
  aarray[katze]=poldi
  echo ${aarray[hund]}   # waldi
  echo ${aarray[*]}      # waldi poldi
 \end{lstlisting}
\end{frame}


\subsection{Funktionen}
\begin{frame}[fragile]
 \frametitle{Funktionen}
 
 \begin{lstlisting}
  func() {
  echo 'Ich bin eine Funktion!'
  echo "Mir wurden folgende Argumente uebergeben: $*"
  local x=foo
  echo $x
  return 4
  echo "Auf mich hoert ja eh keiner"
  }
  
  func a b c
 \end{lstlisting}
\end{frame}

\subsection{Einige wichtige Programme \& Praktischer Teil II}
\begin{frame}[<+->]
 \frametitle{Einige wichtige Programme II}
 \begin{itemize}
  \item {\tt read varname} -- Eingabe von Stdin in Variable speichern
  \item {\tt shift} -- Argumente um ein verschieben
  \item {\tt sleep <n>} -- n Sekunden warten
  \item {\tt head / tail} -- Anfang / Ende von Datei/Stream zeigen
  \item {\tt uniq} -- Doppelte Zeilen entfernen
  \item {\tt printf}
  \item {\tt cut -f<n>} -- nte Spalte augeben
   \begin{itemize}
    \item {\tt -d<x>} -- x als Spaltentrenner
   \end{itemize}
  \item {\tt wc} -- Zeichen (-m), Wörter (-w) oder Zeilen (-l) zählen
  \item {\tt find} -- Dateien finden (Unzählige Optionen!)
 \end{itemize}
\end{frame}

\begin{frame}
 \frametitle{Zweite Ausprobier-Runde}
 \begin{center}
 Los!
 \end{center}
\end{frame}

\section{Fortgeschrittenes}
\subsection{Sed \& Awk}
\begin{frame}
 \frametitle{Sed}
 Stream-Editor zum filtern und bearbeiten von Text
 
 \begin{itemize}
  \item Argumentrenner ist beliebig, üblich: /
  \item Ersetzen aller Stellen von 'alt' durch 'neu': \\ {\tt sed 's/alt/neu/g'}
  \item Ersetze 'a' durch 'x' und 'b' durch 'y': {\tt sed 'y/ab/xy/'}
 \end{itemize}
\end{frame}

\begin{frame}
 \frametitle{Awk}
 \begin{block}{Awk}
  Eine eigene Skriptsprache zur Bearbeitung und Auswertung strukturierter Textdaten!
 \end{block}

  Würde Rahmen dieses Workshops sprengen, wird aber gerne in Shell-Skripten verwendet. 
  \bigskip
  
  Beispiel: \\  
  {\tt echo a,b,c | awk -F, '\{print \$1, \$2\}' }
\end{frame}

\subsection{Prozesssteuerung}
\begin{frame}
 \frametitle{Prozesssteuerung}
 \begin{itemize}
  \item Prozesse im Hintergrund laufend starten: \& am Ende der Zeile \\
    {\tt sleep 999 \& }
  \item Prozessen in den Vordergrund holen: {\tt fg [<n>] }
  \item Prozess beenden: {\tt kill <n>}
 \end{itemize}
\end{frame}

\subsection{Mehr Bash}
\begin{frame}[fragile]
 \frametitle{eval}
 \begin{lstlisting}
  a=b
  b=c
  echo $$a        # 19082b
  echo ${$a}      # "Falsche Variablenersetzung."
  echo \$$a       # $a
  eval echo \$$a  # c 
 \end{lstlisting}
\end{frame}

%% besseres Beispiel, Erklärung?
\begin{frame}[fragile]
 \frametitle{IFS}
 \begin{lstlisting}
 f() { echo $2 ;}
 x=a@b@c
 f $x     # kein output
 IFS=@
 f $x     # b
 \end{lstlisting}
 
 Beispiel:
 \begin{lstlisting} 
 rdom() { local IFS=">"; read -d "<" T C ; }
 
 while rdom; do
     if [ "$T" = "H1" ] ; then
                  echo Ueberschrift: $C
     fi;
 done < $tempfile
 \end{lstlisting}

 
\end{frame}

\begin{frame}
 \frametitle{Parameter Expansion}
 
 Liefern den Wert der Variable…
 \begin{itemize}
  \item {\tt \$\{varname:-alternativ\} } -- oder falls diese leer 'alternativ'
  \item {\tt \$\{varname:offset:length\} } -- auf den Bereich zugeschnitten
  \item {\tt \$\{parameter/alt/neu\} } -- nach Zeichenersetzung
  \item {\tt \$\{\#parameter\} } -- die Länge des Werts der Variable
  \item uvm
 \end{itemize}
\end{frame}

\begin{frame}[fragile]
 \frametitle{Trap}
 Abfangen von Signalen wie durch Ctr+C oder Kill um vor Beendigung noch Tätigkeiten auszuführen.
 \bigskip
 
 \begin{lstlisting}
  trap "echo You pressed Ctr+C!" SIGINT 
 \end{lstlisting}
\end{frame}

\begin{frame}
 \frametitle{Debugging}
 \begin{itemize}
  \item Printf-Debugging: {\tt echo \$var }
  \item {\tt set -x }
  \item Traps
 \end{itemize}
\end{frame}

\begin{frame}
 \frametitle{Dritte Ausprobier-Runde}
 \begin{center}
 Los!
 \end{center}
\end{frame}

\end{document}
